\documentclass[12pt, letterpaper]{article}
\usepackage{parskip}
\usepackage{amsmath}
\usepackage{amssymb}
\usepackage{graphicx}
\usepackage{geometry}
\usepackage{setspace}
\usepackage{subcaption}
\usepackage{anyfontsize}
\usepackage{indentfirst}
\usepackage[utf8]{inputenc}
\usepackage[american]{babel}
\usepackage[babel]{csquotes}
\usepackage[	style=phys,
				articletitle=false,
				biblabel=brackets,
				chaptertitle=false,
				pageranges=false]{biblatex}

\addbibresource{physics-ia.bib}
\defbibheading{bibliography}{\section{Bibliography}}
\bibliography{physics-ia}

\geometry{letterpaper, portrait, margin=1in}
\graphicspath{{../imgs/}}

\newcommand{\sorta}[1]{`#1'}
\newcommand{\poses}[1]{#1's}

\title{The Optimal Strength-Retaining Hole Pattern for Sheet Material}
\author{Tynan Purdy}
\date{May 2019}

\begin{document}
\large
\doublespace{}
\parindent=0.5in

{\fontsize{12}{14.4}
  {\singlespace
    \pagenumbering{gobble}
    \maketitle
    \begin{center}
    002129-0004 \\
    \vspace{4mm}
    IB Physics HL IA \\
    \vspace{4mm}
    Words:  \\ % words
    \end{center}
  }
}	

\newpage
\pagenumbering{arabic}

%TC:break Abstract
\begin{abstract}
One of the greatest challenges of structural engineering is to reduce the weight of a system without compromising its strength. Hole patterns are a go-to solution to make parts lighter and maintain the majority of their rigidity. The problem is, what hole pattern is best? With many 2D tessellation patterns to choose from, it can be difficult to determine the optimal pattern to use. This investigation will simulate stresses on test parts with a variety of hole patterns to determine which shape maintains the highest strength in an array of scenarios. 

Words: % words

\end{abstract}
%TC:break _main_

\newpage
\tableofcontents
\newpage

\section{Background}



\section{Simulation}

\subsection{Variables}

\subsection{Procedure}

\section{Analysis}

\section{Conclusion}

\newpage
\printbibliography{}

\newpage
\section{Appendix}
\listoffigures{}

\end{document}